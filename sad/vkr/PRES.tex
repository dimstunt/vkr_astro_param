\documentclass{beamer}
\usepackage[utf8]{inputenc} 
\usepackage[russian]{babel} 
\newenvironment{compactlist}{
    \begin{list}{{$\bullet$}}{
      \setlength\partopsep{0pt}
      \setlength\parskip{0pt}
      \setlength\parsep{0pt}
      \setlength\topsep{0pt}
      \setlength\itemsep{0pt}
} }{
\end{list} }
\begin{document}
	\title{Параметризация звезд и межзвездного вещества на основе решения обратной задачи классификации над данными наблюдений}   
	\author{Черкашин Дмитрий} 
	\date{} 
	
	
	\frame{\titlepage
	\begin{center}
		Научный руководитель -- Скворцов Н.А.
	\end{center}
	}
	
	\frame{\frametitle{Содержание}\tableofcontents}
	\section{Введение}
	\frame{\frametitle{Введение}
		Одна из основных проблем астрофизики - изучение физических свойств, принадлежащих поверхностным слоям звезд (температура, гравитация, металличность,...)
		
		
		Звезды имеют достаточно четкую классификацию с известными параметрами --> зная класс звезды можно определить ее физические свойства 


	}
	\frame{ 
		\frametitle{Межвездное поглощение}
		Звезды наблюдаются сквозь межзвездную пыль --> излучение тускнеет и краснеет --> проблемы с классификацией
		\newline
		\newline
		\newline
		Межзвездное поглощение - суммарный эффект рассеивания и истинного поглощения электромагнитного излучения пылью и газом. Зная коэффициент м.п. можно правильно определить изначальное излучение 
	}



	\frame{ 
		\frametitle{Способы параметризации}
		1. Использование большого телескопа в течение продолжительного времени
		
		2. Построение эволюционных треков (так же требует длительного наблюдения)
		
		3. Построение карты межзвездного поглощения на основе корреляции между плотностью пылевого столба и распределением нейтрального водорода
		
		4. Решение обратной задачи спектрального анализа
	}
	
	\section{Постановка}
	\frame{\frametitle{Постановка} 
		Цель: разработка подходов и средств распределенного анализа данных для параметризации звезд на основе решения обратной задачи классификации над многоцветными фотометрическими обзорами звездного неба.
		\newline
			}


	\frame{\frametitle{Задачи} 
		\begin{compactlist}
		\item Разработка концептуального представления фотометрических данных звёзд и отображение в него схем оригинальных каталогов для трансформации данных множественных наблюдений звёзд в общее представление.
		\item Разработка подхода к организации доступа и разрешению сущностей (перекрёстному отождествлению) среди множественных наблюдений звёзд в неоднородных данных обзоров неба.
		\item Модификация алгоритма параметризации звезд на основе методов решения обратных задач
		\item Разработка параллельной распределённой реализации предложенных подходов
	\end{compactlist}
		}
		

	\section{Обратные задачи в астрофизике}
	\frame{ 
		\frametitle{Обратные задачи в астрофизике}
		Прямая задача: нахождение следствия некоторого процесса по известной исследователю причине
		\newline
		Обратная задача: один и тот же эффект может быть порожден разными причинами (пример - кипение воды)
	}
	
	\frame{ 
		\frametitle{Корректность обратной задачи}
		Задача корректна, если
		\begin{compactlist}
		\item ее решение существует
		\item решение единственно
		\item решение непрерывно зависит от входных данных (условие устойчивости решения)
		\end{compactlist}
	}

	\frame{ 
		\frametitle{Математическая формулировка}
		$$
		u(x)=A(x,z(s))
		$$
		$A$ – оператор, устанавливающий причинно-следственную связь между $z(s)$ и $u(x)$.
		\newline
		Либо уравнение Фредгольдма 1-го рода
		\newline
		$$
		u(x) = \int^a_b K(x, s)z(s) ds
		$$
		где $K(x, s)$ – ядро (непрерывное или квадратично суммируемое по переменным $x$, $s$), которое описывает конкретную модель исследуемого процесса.
	}

	\frame{\frametitle{Метод решения некорректно поставленных задач, предложенный А.Н. Тихоновым} 
		Некорректные задачи нужно доопределить. Для этого необходима дополнительная (априорная) информация об искомом решении $z(s)$, вытекающая из обширного опыта всесторонних исследований данного процесса.
		
		Такой информацией могут служить сведения о гладкости искомого решения $z(s)$, его монотонности, выпуклости, неотрицательности, принадлежности к конечно-параметрическому семейству и т. п.
	}

				
	\section{Интеграция обзоров звездного неба}
	\frame{\frametitle{Интеграция обзоров звездного неба}
	DENIS - 135.677 звезд 
	
	ALLWISE - 1.169.013 звезд
	\begin{figure}[h]
	\centering
	\includegraphics[width=10cm]{dds.png}
	\end{figure}
	}
	
	
	

	\frame{\frametitle{Кросс-отожествление обзоров звездного неба}
		Методы кросс-отожествления
		\begin{compactlist}
		\item Структурный
		\item Координатный
		\item Координатное сопоставление с фильтрацией объектов
		\end{compactlist}
	}

	\frame{\frametitle{Реализация}
		Пакет AstroML для Spark
		
		112.531 успешных сопоставлений
		\begin{figure}[h]
		\centering
		\includegraphics[width=7cm]{lol}
		\caption{Гистограмма расстояния между координатами в ALLWISE и DENIS}
	\end{figure}
	}
	
	\section{Алгоритм обратного спектрального анализа}
	\frame{\frametitle{Алгоритм обратного спектрального анализа}	
		Для нахождения спектрального типа $SpT$, расстояния до звезды $d$ и межзвездное поглощение $A_V$, необходимо минимизировать функционал
	$$
	D^2 = \sum_{i=1}^N \left(\frac{m_{obs,i}-m_{calc,i}}{\Delta m_{obs,i}} \right)^2
	$$
	где суммирование происходит по всем известным диапазонам ($N = 7$), и
	$m_{calc,i} = M_i(SpT) + 5 \log d - 5 + A_i(A_V)$

	}
	\frame{\frametitle{Алгоритм обратного спектрального анализа}	
		
	$A_i(A_V) = k_i*A_V$ - закон межзвездного поглощения для i-той фотометрической полосы.
	$k_i$ - коэффициент затухания блеска для i-той фотометрической полосы.
	\begin{table}[!ht]
	\centering
	\setlength{\tabcolsep}{3pt}
	\begin{tabular}{|l|l|l|l|l|l|l|l|}
	\hline
	Длина волны & I & J & K & W1 & W2 & W3 & W4 \\ \hline
	Коэффициент & 1.71 & 0.72 & 0.30 & 0.18 & 0.16 & 0.14 & 0.11 \\ \hline
	\end{tabular}
	\caption{Коэффициенты затухания блеска}
	\end{table}

	}
	\frame{\frametitle{Алгоритм обратного спектрального анализа}	
		$M_i(SpT)$ - абсолютный блеск в i-той фотометрической полосе. Необходимы абсолютные величины звезд разных спектральных типов в соответствующих фотометрических системах.
		\newline
	\begin{table}
	\centering
	\small
	\setlength{\tabcolsep}{3pt}
	\begin{tabular}{|l|l|l|l|l|l|l|l|l|l|}
	\hline
	SpT & MB    & MR    & MI    & MJ    & MK    & MW1   & MW2   & MW3   & MW4   \\ \hline
	B8  & 0.32  & 0.39  & 0.04  & 0.34  & 0.62  & 0.01  & 0.10  & 0.11  & 1.00  \\ \hline
	A0  & 1.58  & 0.47  & 0.72  & 1.04  & 1.28  & 0.54  & 0.58  & 0.56  & 0.30  \\ \hline
	..  & ..    & ..    & ..    & ..    & ..    & ..    & ..    & ..    & ..    \\ \hline
	\end{tabular}
	\caption{Абсолютные значения видимой звездной величины при различных спектральных типах звезд}
	\end{table}


	}
	\frame{\frametitle{Алгоритм обратного спектрального анализа}	
	$A_V$ - коэффициент преломления для звезды, находящейся на расстоянии $d$, вычисляется по формуле (аналог земного бараметрического закона):
	$$
	A_V = \frac{a_O \beta}{\sin |b|} \left( 1 - e^{\frac{-d \sin |b|}{\beta}}\right)
	$$
	$a_O$ - коэффициент поглощения (параметр),
	$\beta$ - коэффициент вертикального поглощения преломляющегося света,
	$b$ - галактическая долгота
	}
		\section{Методы минимизации функции нескольких переменных}
	\frame{\frametitle{метод градиентного спуска}
		$$
		x^{i+1}_j = x^i_j - h * \frac{df}{dx^i_j} = x^i_j - h * \frac{df/dx^i_j}{|\nabla f(x^i)|}, j = 1,...,n
		$$
		
		2 этапа:
		\begin{compactlist}
			\item Оценка градиента $F(x)$ путем вычисления частных производных от $F(x)$ по каждой переменной $x_j$
			\item Рабочий шаг по всем переменным одновременно
		\end{compactlist}

		Для оценки частных производных используются разностные методы:
	\begin{compactlist}
		\item Алгоритм с центральной пробой: $\frac{df}{dx_i} \approx 
		\frac{f(x_1,...,x_i+g_i,...,x_n)-f(x_1,...,x_i,...,x_n)}{g_i}$	
		\item Алгоритм с парными пробами: $\frac{df}{dx_i} \approx 
		\frac{f(x_1,...,x_i+g_i,...,x_n)-f(x_1,...,x_i-g_i,...,x_n)}{g_i}$	
	\end{compactlist}
	}
	\frame{\frametitle{метод скорейшего спуска}
		В текущей точке вычисляется  $\nabla f(x)$, и затем в направлении градиента ищется $\min f(x)$.
		
		Вдали от оптимума эффективность метода повышается.
	}

	\frame{\frametitle{Реализация}	
	Алгоритм реализован на Spark. Для каждой звезды перебирая спектральный тип ищем минимум функционала по расстоянию и коэффициентам поглощения, затем отбираем минимальный, репартиционируем, сохраняем полученный тип
	
	В результате получаем связку <<координаты -- класс звезды -- расстояние до звезды -- коэффициент поглощения -- коэффициент вертикального поглощения преломляющегося света>>
	\begin{table}
	\centering
	\small
	\setlength{\tabcolsep}{3pt}
	\begin{tabular}{|l|l|l|l|l|l|} \hline
	   RAJ2000 &   DEJ2000 & SpT &  d &  a$\_$0 & beta \\ \hline
	130.895126 &  1.711113 &  B8 & 30 & 0.1 &   30 \\ \hline
	 134.26753 & -1.468054 &  B8 & 1000 & 0.01 &  270 \\ \hline
	 136.378296 & 1.129839 & F0 & 120 & 0.06 &  30 \\ \hline
	 ... & ... & ... & ... & ... & ... \\ \hline
	\end{tabular}
	\caption{Абсолютные значения видимой звездной величины при различных спектральных типах звезд}
	\end{table}
	
	}
	
	\frame{\frametitle{Дальнейшая работа}	
		\begin{compactlist}
		\item Добавить новые источники (Gaia DR2, IPHAS DR2, LAMOST и др.)
		\item Реализовать модификацию перекрестного отожествления звезд с учетом фильтрации объектов
		\item Доработать алгоритм обратного спектрального анализа таким образом, что бы при минимизации функционала учитывались значения полученного коэффициента поглощения и расстояния для соседних звезд, воспользоваться другими подходами к оптимизации процедуры и точности определения параметров.
		\item Оптимизация решения задачи
		\end{compactlist}
}
\end{document}
